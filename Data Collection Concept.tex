\documentclass[12pt,a4paper,Data Collection Cencept]{report}

\usepackage[utf8]{inputenc}

\usepackage{amsmath}

\usepackage{amsfonts}

\usepackage{amssymb}

\usepackage{graphicx}

\begin{document}


%\selectlanguage{english} %%% remove comment delimiter ('%') and select language if required



\noindent \textbf{DATA COLLECTION CONCEPT}

\\I have set up a data collection system using the Google Appengine in connection to ODK Aggregate Server and ODK Collect. In this complain, I aim to collect information about the beautiful scenes that people might witness and want to share with others or keep as ‘trophies’. They can take pictures, capture videos, write comments or descriptions and take location of the site. Information can include capture of moments in action, documentaries, events or nature. There is pretty much we can do with this data.
\\

\noindent\textbf{SAMPLE DATA}

\\ 

\begin{figure}

\includegraphics[width\linewidth{kyeswaData.PNG} 

\caption{Sample data.}

\label{fig: Sample data}
 
\end{figure} 

Figure \ref{fig:Sample data} shows a Sample data.
\noindent \textbf{WHAT PROBLEM THE DATA CAN SOLVE}
\noindent 


\begin{enumerate}

\item \textbf{This data can be solid proof during criminal investigation investigation especially when is was been sorted properly and grouped according to priority and example is Sselubogo Moses' dramatic attack and beat-up.}


\item \textbf{This data can be broadcasted for advertising tourism sites. It is well structured in such a way that it has all the necessary details for intrested tourists.}


\item \textbf{This information can also be used for field work evaluation using distributed teams for example door-to-do evangelism. After the day's activity, the teams can gather up to evaluate work span and its effect.
}

\end{enumerate}

\noindent

\end{document}

